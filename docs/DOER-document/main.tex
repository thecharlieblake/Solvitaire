\documentclass{article}
\usepackage[utf8]{inputenc}
\usepackage[margin=1in]{geometry}
\usepackage{hyperref}
\usepackage[
	backend=biber,
	style=alphabetic,
	sorting=ynt
]{biblatex}
\addbibresource{references.bib}

\title{Description, Objectives, Ethics, Resources}
\author{Matriculation Number: 150012297}
\date{September 2017}

\begin{document}

\maketitle

\section{Description}

\subsection{Title}

\vspace{2mm}

\emph{\large{A Solver for Solitaire Games of Perfect Information}}

\subsection{Overview}

The aim of this project is to create a single solver for a variety of
single-player, perfect information solitaire card games. The solver should be
able to correctly provide solutions for some instances of solitaire games, and
be able to report other instances as unsolvable, minimising the number of
instances that are deemed 'intractable' (cannot be classified as solvable or
unsolvable within a reasonable time period). A key study for this project will
be how to do this efficiently for a \emph{variety} of solitaire games.

Perfect information games are an especially important domain in artificial
intelligence, and some of the most high-profile successes in the field of AI
have been creating good players for games like Chess, Checkers and Go, all of
which are perfect information games. However, relatively little work has been
done on creating programs to play single-player perfect information games, and
good solvers do not yet exist for most solitaire games.

Hence, this project will focus on creating a solver which is general enough to
be able to play a wide range of solitaire games, whilst also maintaining a high
level of efficiency. This efficiency will be achieved through a combination of
language-level optimisation, and through exploring different AI search
techniques. A specialised solver will be written to accomplish this, in code
rather than using an existing constraint solving or planning programs.

\section{Objectives}

\subsection{Primary Objectives}

\begin{enumerate}
	\item An application for solving single-player, perfect information solitaire
		instances (software artefact).
  \item The application must be able to solve a \emph{wide} range of games.
  \item The solver's classifications must be \emph{demonstrably} accurate, with
		a reasonably low rate of intractable (cannot be classified as solvable or
		unsolvable within a sensible time period) instances.
  \item Some kind of schema or language for describing solitaire games.
	\item An analysis of the effects of different search techniques and
		implementation choices on the performance of the solver.
	\item An implementation of `meta-moves'/`super moves'\cite{SolLab} for the
		solver's search, for those games in which it is applicable.
\end{enumerate}

\subsection{Secondary Objectives}

\begin{enumerate}
	\item Optimise the search algorithm to eliminate searching states in which
		certain moves/states `dominate'\cite{WikDom} others.
	\item Implement more than one different type of search (e.g. best first,
		iterative deepening).
	\item Design a general heuristic for best-first search that works across a
		variety of games.
\end{enumerate}

\subsection{Tertiary Objectives}

Other tertiary objectives that satisfy the core aims of the project would be
reasonable additions. Below are two particular instances of interesting routes
down which this project could be taken:

\begin{enumerate}
	\item An extension of the algorithm that aims to provide shortened solutions
		(i.e. ones that minimise the number of moves taken).
	\item An implementation of Monte-Carlo tree search\cite{WikMCTS} to find
		solutions.
\end{enumerate}

\section{Ethics}

There are no ethical considerations for this project.

\section{Resources}

This project can be completed using standard school equipment.

\printbibliography

\end{document}
